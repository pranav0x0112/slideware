\documentclass{beamer}
\usetheme{Madrid}
\setbeamertemplate{footline}{
  \hfill
  \insertframenumber{} / \inserttotalframenumber
  \hspace{0.5cm}
}
\setbeamertemplate{navigation symbols}{}
\setbeamertemplate{headline}{
  \begin{beamercolorbox}[wd=\paperwidth,ht=11.5ex,dp=1.5ex]{}
    \hfill
    \includegraphics[height=1cm]{images/chips_logo.png}
    \hspace{0.3cm}
  \end{beamercolorbox}
}

\usepackage{graphicx}

\title{SDR, DDR \& LPDDR}
\subtitle{Evolution of Synchronous DRAM Interfaces}
\author{
Pranav M -- PES2UG23EC076\\
Lalith -- PES2UG23EC074\\
Harshini -- PES2UG23EC058
}
\begin{document}
\date{}

%------------------------------------------------
% Slide 1: Title
%------------------------------------------------
\begin{frame}
    \titlepage
\end{frame}


% Slide 2: Where DRAM sits in a system
\begin{frame}{Where DRAM sits in a system}
    \begin{itemize}
        \item The CPU executes instructions and requests data continuously
        \item DRAM acts as the main working memory for active programs and data
        \item Storage is used for long-term data and is accessed much less frequently
    \end{itemize}

    \vspace{0.3cm}
    \centering
    \includegraphics[width=0.8\textwidth]{images/system_view.png}
\end{frame}


% Slide 3: Why DRAM exists
\begin{frame}{Why DRAM exists}
    \begin{itemize}
        \item Accessing storage directly is too slow for CPU execution
        \item Cache memory is fast but limited in size and expensive
        \item DRAM provides a balance between speed, capacity, and cost
    \end{itemize}

    \vspace{0.3cm}
    \centering
    \includegraphics[width=0.6\textwidth]{images/dram_tradeoff.png}
\end{frame}

% Slide 4: What is SDRAM
\begin{frame}{What is SDRAM}
    \begin{itemize}
        \item SDRAM stands for Dynamic Random Access Memory
        \item All operations are synchronized to an external clock
        \item Data is stored as charge in capacitor-based memory cells
    \end{itemize}

    \vspace{0.3cm}
    \centering
    \includegraphics[width=0.4\textwidth]{images/sdram_basic.png}
\end{frame}


% Slide 5: Working of SDRAM
\begin{frame}{Working of SDRAM}
    \begin{itemize}
        \item A row is activated, loading data into the sense amplifiers
        \item A column is selected to perform a read or write operation
        \item The row is precharged before accessing the next row
    \end{itemize}

    \vspace{0.3cm}
    \centering
    \includegraphics[width=0.4\textwidth]{images/sdram_working.png}
\end{frame}

% Slide 6: Read and Write Operations in SDRAM
\begin{frame}{Read and Write Operations in SDRAM}
    \begin{itemize}
        \item Writing stores charge in a capacitor through an access transistor
        \item Reading transfers the stored charge to the bitline
        \item Sense amplifiers detect and amplify very small voltage changes
        \item Read operations are destructive and require immediate restoration
    \end{itemize}
\end{frame}

% Slide 7: Refresh Operation in SDRAM
\begin{frame}{Refresh Operation in SDRAM}
    \begin{itemize}
        \item Charge stored in capacitors leaks over time
        \item SDRAM periodically refreshes all rows to retain data
        \item Refresh internally reads and restores stored charge
        \item Typical refresh interval is around 64 ms
    \end{itemize}
\end{frame}

% Slide 8: Memory Organization in SDRAM

\begin{frame}{Memory Organization in SDRAM}
    \begin{itemize}
        \item Unlike SRAM, SDRAM is not a simple linear array
        \item Memory addresses are divided into bank, row, and column fields
        \item Address format: [Bank][Row][Column]
        \item This organization reflects the physical layout of DRAM cells
    \end{itemize}

    \vspace{0.3cm}
    \centering
    \includegraphics[width=0.3\textwidth]{images/sdram_addressing.png}
\end{frame}

%------------------------------------------------
% Slide 9: SDRAM Internal Block Diagram
%------------------------------------------------
\begin{frame}{SDRAM Internal Block Diagram (Architecture View)}
    \centering
    \includegraphics[width=0.7\textwidth]{images/sdram_block_diagram.png}
\end{frame}

%------------------------------------------------
% Slide 10: Why Banks, Rows, and Columns?
%------------------------------------------------
\begin{frame}{Why Banks, Rows, and Columns?}
    \begin{itemize}
        \item DRAM cells are arranged as a 2D array for efficient fabrication
        \item Entire rows are sensed together using shared sense amplifiers
        \item Keeping a row open enables fast repeated accesses (row buffer hits)
        \item Multiple banks allow parallel operations and hide memory latency
    \end{itemize}

    \vspace{0.3cm}
    \centering
    \includegraphics[width=0.5\textwidth]{images/sdram_banks.png}
\end{frame}

%------------------------------------------------
% Slide 11: SDRAM Timing Diagram
%------------------------------------------------
\begin{frame}{SDRAM Timing: ACT $\rightarrow$ READ $\rightarrow$ PRE}
    \centering
    \includegraphics[width=0.7\textwidth]{images/sdram_timing.png}
\end{frame}

%------------------------------------------------
% Slide 12: SDR (Single Data Rate)
%------------------------------------------------
\begin{frame}{SDR (Single Data Rate)}
    \begin{itemize}
        \item Transfers one data word per clock cycle
        \item Data is sampled only on the rising edge of the clock
        \item Memory bandwidth scales linearly with clock frequency (one transfer per cycle)
        \item Row and column commands are synchronized to the clock
        \item Internal DRAM core operates asynchronously but is synchronized at the interface
    \end{itemize}

    \vspace{0.3cm}
    \centering
    \includegraphics[width=0.6\textwidth]{images/sdr_timing.png}
\end{frame}

\begin{frame}{SDR SDRAM Internal Block Diagram (Architecture View)}
    \centering
    \includegraphics[width=0.7\textwidth]{images/sdr_sdram_block_diagram.png}
\end{frame}

%------------------------------------------------
% Slide 13: SDR Key Operations
%------------------------------------------------
\begin{frame}{SDR Key Operations}
    \begin{itemize}
        \item RAS command activates a row in the selected bank
        \item Activated row data is latched into sense amplifiers (row buffer)
        \item Column READ or WRITE accesses data from the active row buffer
        \item Precharge closes the row before another row can be activated
    \end{itemize}

    \vspace{0.3cm}
    \centering
    \includegraphics[height=0.4\textheight]{images/sdr_operations.png}
\end{frame}

%------------------------------------------------
% Slide 14: SDR Timing Parameters and Commands
%------------------------------------------------
\begin{frame}{SDR Timing Parameters and Commands}
    \begin{itemize}
        \item tRCD: Delay between row activation and column access
        \item tRP: Time required to precharge before next activation
        \item CL (CAS Latency): Cycles from READ command to data output
        \item All operations are constrained by strict timing parameters
    \end{itemize}

    \vspace{0.3cm}
    \centering
    \includegraphics[width=0.5\textwidth]{images/sdr_timing_params.png}
\end{frame}

%------------------------------------------------
% Slide 15: Why Use SDR SDRAM?
%------------------------------------------------
\begin{frame}{Why Use SDR SDRAM?}
    \begin{itemize}
        \item Cost-effective, high-density memory compared to SRAM
        \item Sufficient bandwidth for embedded and legacy systems
        \item Well-standardized with simple controller design
        \item Serves as a foundation for understanding DDR technologies
    \end{itemize}
\end{frame}

%------------------------------------------------
% Slide 14: DDR (Double Data Rate)
%------------------------------------------------
\begin{frame}{DDR (Double Data Rate)}
    \begin{itemize}
        \item Transfers two data words per clock cycle
        \item Data is sampled on both rising and falling clock edges
        \item Achieves higher bandwidth without increasing clock frequency
    \end{itemize}

    \vspace{0.3cm}
    \centering
    \includegraphics[width=0.5\textwidth]{images/ddr_timing.png}
\end{frame}

%------------------------------------------------
% Slide 15: DDR Architectural Upgrade (2n Prefetch)
%------------------------------------------------
\begin{frame}{DDR Architectural Upgrade: 2n Prefetch}
    \begin{itemize}
        \item DRAM core operates at single-edge (SDR-like) speed
        \item Prefetch buffer reads multiple data words per access
        \item I/O interface transmits data on both clock edges
        \item Enables high bandwidth without faster memory cells
    \end{itemize}

    \vspace{0.3cm}
    \centering
    \includegraphics[width=0.5\textwidth]{images/ddr_prefetch.png}
\end{frame}


%------------------------------------------------
% Slide 16: Clocking and Data Capture in DDR
%------------------------------------------------
\begin{frame}{Clocking and Data Capture in DDR}
    \begin{itemize}
        \item Uses differential clock signals (CK and CK\#)
        \item Introduces Data Strobe (DQS) for precise data capture
        \item Delay-Locked Loop (DLL) aligns internal and external timing
        \item Improves signal integrity at higher operating frequencies
    \end{itemize}
\end{frame}


%------------------------------------------------
% Slide 17: DDR Generations and Scaling
%------------------------------------------------
\begin{frame}{DDR Generations and Scaling}
    \begin{itemize}
        \item Prefetch depth increases across generations
        \item External I/O bandwidth scales faster than core speed
        \item Bank groups introduced to sustain high throughput
        \item Operating voltage reduces with each generation
    \end{itemize}

    \vspace{0.3cm}
    \centering
    \includegraphics[width=0.5\textwidth]{images/ddr_generations.png}
\end{frame}


%------------------------------------------------
% Slide 18: Advantages and Trade-offs of DDR
%------------------------------------------------
\begin{frame}{Advantages and Trade-offs of DDR}
    \begin{itemize}
        \item Doubles bandwidth without increasing clock frequency
        \item Prefetch and bank interleaving improve throughput
        \item Efficient for sequential access patterns
        \item Trade-off: higher latency for random accesses
    \end{itemize}

    \vspace{0.3cm}
    \centering
\end{frame}

%------------------------------------------------
% Slide 19: What is LPDDR?
%------------------------------------------------
\begin{frame}{LPDDR: Low Power DDR}
    \begin{itemize}
        \item Designed for mobile and battery-powered devices
        \item Prioritizes power efficiency over peak performance
        \item Based on DDR architecture with aggressive power optimizations
        \item Widely used with SoCs in phones, tablets, and embedded systems
    \end{itemize}

    \vspace{0.3cm}
    \centering
    \includegraphics[width=0.6\textwidth]{images/lpddr_overview.png}
\end{frame}

%------------------------------------------------
% Slide 20: Power-Saving Techniques in LPDDR
%------------------------------------------------
\begin{frame}{Power-Saving Techniques in LPDDR}
    \begin{itemize}
        \item Operates at significantly lower core and I/O voltages
        \item Supports multiple low-power and deep sleep states
        \item Self-refresh and deep power-down reduce idle power
        \item Optimized refresh and clock gating minimize energy usage
    \end{itemize}

    \vspace{0.3cm}
    \centering
    \includegraphics[height=0.5\textheight]{images/lpddr_power_states.png}
\end{frame}


%------------------------------------------------
% Slide 21: Evolution of LPDDR
%------------------------------------------------
\begin{frame}{Evolution of LPDDR}
    \begin{itemize}
        \item LPDDR generations steadily reduce operating voltage
        \item Bandwidth increases while maintaining power efficiency
        \item Prefetch depth increases across generations
        \item New features added for reliability and energy savings
    \end{itemize}

    \vspace{0.3cm}
    \centering
    \includegraphics[width=0.7\textwidth]{images/lpddr_generations.png}
\end{frame}

%------------------------------------------------
% Slide 22: SDR vs DDR vs LPDDR
%------------------------------------------------
\begin{frame}{SDR vs DDR vs LPDDR}
    \begin{itemize}
        \item SDR: Simple design with limited bandwidth
        \item DDR: High performance through double-edge transfers
        \item LPDDR: Optimized for low power and mobile systems
        \item Choice depends on performance, power, and system needs
    \end{itemize}

    \vspace{0.3cm}
    \centering
    % \includegraphics[width=0.7\textwidth]{images/memory_comparison.png}
\end{frame}

%------------------------------------------------
% Slide 23: Key Takeaways
%------------------------------------------------
\begin{frame}{Key Takeaways}
    \begin{itemize}
        \item All DRAM types share the same core memory principles
        \item SDR, DDR, and LPDDR differ in interface and optimization goals
        \item Modern systems choose memory based on performance and power needs
    \end{itemize}

    \vspace{0.3cm}
    \centering
    % \includegraphics[width=0.5\textwidth]{images/final_tradeoff.png}
\end{frame}


\end{document}