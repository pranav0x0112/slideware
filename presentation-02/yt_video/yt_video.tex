\documentclass{beamer}
\usetheme{default}
\usecolortheme{default}
\usepackage{graphicx}
\date{}
\setbeamertemplate{navigation symbols}{}
\setbeamertemplate{headline}{
  \begin{beamercolorbox}[wd=\paperwidth,ht=11.5ex,dp=1.5ex]{}
    \hfill
    \includegraphics[height=1cm]{../images_yt_video/chips_logo.png}
    \hspace{0.3cm}
  \end{beamercolorbox}
}
% Footline: show slide number and total slides
\setbeamertemplate{footline}{%
    \begin{beamercolorbox}[wd=\paperwidth,ht=2.5ex,dp=2.5ex]{}%
       \makebox[\paperwidth][r]{\kern-0.7cm Slide \insertframenumber{} / \inserttotalframenumber\hspace{0.3cm}}%
    \end{beamercolorbox}%
}
\title{DRAM and DDR Memory}
\subtitle{Dynamic Random Access Memory Architecture}

\begin{document}

\frame{\titlepage}

%% Slide 2: How They Work Together
\begin{frame}{Loading and Prefetching}
\textbf{Loading:}
\begin{itemize}
    \item Data copied from SSD to DRAM during program startup
    \item CPU can only process data after it's in DRAM
    \item Takes time (loading bar appears)
\end{itemize}

\vspace{0.5cm}
\textbf{Prefetching:}
\begin{itemize}
    \item Moving data to DRAM before it's needed
    \item Anticipates future data requirements
    \item Minimizes wait time during execution
\end{itemize}
\end{frame}

%% Slide 3: DRAM Module Overview
\begin{frame}{DRAM Module Overview (DIMM)}
\textbf{Structure:}
\begin{itemize}
    \item DIMM = Dual Inline Memory Module
    \item 8 DRAM chips per module (typical)
    \item Connected to CPU via memory channels
\end{itemize}

\vspace{0.5cm}
\textbf{Connection:}
\begin{itemize}
    \item Two independent memory channels
    \item Memory controller in CPU manages communication
\end{itemize}
\end{frame}

%% Slide 4: Memory Channels (DDR5)
\begin{frame}{Memory Channels in DDR5}
\textbf{Channel Structure:}
\begin{itemize}
    \item Each memory channel divided into Channel A and Channel B
    \item Independent operation, 32 bits transferred per channel
\end{itemize}

\vspace{0.3cm}
\textbf{Signal Lines per Channel:}
\begin{itemize}
    \item 32 data wires for actual data transfer
    \item 21 address wires for memory location
    \item 7 control wires for commands
\end{itemize}

\vspace{0.3cm}
\textbf{Parallel Processing:}
\begin{itemize}
    \item All 4 chips on channel receive same address/commands
    \item Data lines divided: each chip handles 8 bits
    \item Power managed by dedicated chips on module
\end{itemize}
\end{frame}

%% Slide 5: Inside a DRAM Chip
\begin{frame}{Inside a DRAM Chip}
\textbf{Physical Components:}
\begin{itemize}
    \item Packaging contains interconnection matrix
    \item Ball grid array connects to die (main chip)
\end{itemize}

\vspace{0.3cm}
\textbf{Die Organization (2 GB chip):}
\begin{itemize}
    \item 8 bank groups × 4 banks = \textbf{32 banks total}
    \item Each bank: 65,536 rows × 8,192 columns
    \item Total: $\sim$17 billion memory cells
    \item Complex network of wires and supporting circuits
\end{itemize}
\end{frame}

%% Slide 6: DRAM Block Diagram
\begin{frame}
\begin{center}
\includegraphics[width=0.95\textwidth]{../images_yt_video/dram_block_diagram.png}
\end{center}
\end{frame}

%% Slide 7: Memory Addressing Scheme
\begin{frame}{31-Bit Memory Addressing}
\textbf{Accessing 17 billion cells requires 31-bit address:}

\vspace{0.3cm}
\begin{itemize}
    \item \textbf{3 bits:} Bank group selection (8 groups)
    \item \textbf{2 bits:} Bank selection (4 per group)
    \item \textbf{16 bits:} Row selection (65,536 rows)
    \item \textbf{10 bits:} Column group (8,192 ÷ 8)
\end{itemize}

\vspace{0.3cm}
\textbf{Transmission Optimization:}
\begin{itemize}
    \item Address sent in two parts using 21 wires
    \item Part 1: Bank group + bank + row
    \item Part 2: Column address
    \item Each access reads/writes 8 bits simultaneously
\end{itemize}
\end{frame}

%% Slide 8: Memory Cell Structure (1T1C)
\begin{frame}{Memory Cell Structure: 1T1C}
\textbf{Components (each stores 1 bit):}

\vspace{0.3cm}
\textbf{1. Capacitor:}
\begin{itemize}
    \item Deep trench in silicon (few dozen nanometers)
    \item Two conductive surfaces with dielectric insulator
    \item 1V = binary 1, 0V = binary 0
\end{itemize}

\vspace{0.3cm}
\textbf{2. Access Transistor:}
\begin{itemize}
    \item Wordline activates transistor gate
    \item Bitline connects to transistor channel
    \item Controls capacitor access
\end{itemize}
\end{frame}

%% Slide 9: How Memory Cells Store Data
\begin{frame}{How Memory Cells Store Data}
\textbf{Write Operation:}
\begin{itemize}
    \item Wordline ON $\rightarrow$ connects capacitor to bitline
    \item Charge flows to write 1 or discharge for 0
\end{itemize}

\vspace{0.3cm}
\textbf{Read Operation:}
\begin{itemize}
    \item Wordline ON $\rightarrow$ measure capacitor charge
    \item Charge amount indicates stored value
\end{itemize}

\vspace{0.3cm}
\textbf{Data Retention Challenge:}
\begin{itemize}
    \item Wordline OFF isolates capacitor
    \item Electron leakage through tiny transistor
    \item Requires periodic refresh
\end{itemize}
\end{frame}

%% Slide 10: Memory Array Organization
\begin{frame}{Memory Array Organization}
\textbf{Array Structure:}
\begin{itemize}
    \item \textbf{Wordlines:} Rows connecting to transistor gates
    \item \textbf{Bitlines:} Columns connecting to transistor channels
    \item Different vertical layers (no physical contact)
\end{itemize}

\vspace{0.3cm}
\textbf{Operation Rules:}
\begin{itemize}
    \item Active wordline connects entire row to bitlines
    \item \textbf{Only ONE wordline active at a time}
    \item Multiple active wordlines cause data interference
\end{itemize}

\vspace{0.3cm}
\textbf{Cell Access:}
\begin{itemize}
    \item Row decoder: 16 bits activate single wordline
    \item Column multiplexer: 10 bits select 8 bitlines
\end{itemize}
\end{frame}

%% Slide 11: Array Organization Diagram
\begin{frame}
\begin{center}
\includegraphics[width=0.95\textwidth]{../images_yt_video/column_select.png}
\end{center}
\end{frame}

%% Slide 12: Reading from Memory Cells
\begin{frame}{Reading from Memory Cells}
\textbf{Process:}
\begin{itemize}
    \item CPU sends read command + address
    \item Select bank, precharge bitlines to 0.5V
    \item Activate wordline (one row)
    \item Capacitors slightly change bitline voltage
    \item Sense amplifiers amplify to full 1V or 0V
    \item Column multiplexer selects 8 specific bits
    \item Data transmitted to CPU
\end{itemize}
\end{frame}

%% Slide 13: Read Process Diagram
\begin{frame}
\begin{center}
\includegraphics[width=0.95\textwidth]{../images_yt_video/tRAS.png}
\end{center}
\end{frame}

%% Slide 14: Writing to Memory Cells
\begin{frame}{Writing to Memory Cells}
\textbf{Process:}
\begin{itemize}
    \item CPU sends write command + address + data
    \item Select bank, precharge bitlines
    \item Activate row, sense amplifiers open row
    \item Column multiplexer selects 8 bitlines
    \item Write drivers override voltages (1V or 0V)
    \item Capacitors updated with new data
\end{itemize}

\vspace{0.5cm}
\textbf{Note:} All 4 chips operate in parallel
\end{frame}

\begin{frame}
\begin{center}
\includegraphics[width=0.95\textwidth]{../images_yt_video/tRAS_full.png}
\end{center}
\end{frame}

%% Slide 15: Refresh Operation
\begin{frame}{Refresh Operation}
\textbf{Why Refresh:}
\begin{itemize}
    \item Electrons leak through nanoscale transistors
    \item Data loss prevention
\end{itemize}

\vspace{0.5cm}
\textbf{Process:}
\begin{itemize}
    \item Precharge bitlines to 0.5V
    \item Open one row at a time
    \item Sense amplifiers restore full voltage
\end{itemize}

\vspace{0.5cm}
\textbf{Timing:} 50ns per row, occurs every 64ms per bank
\end{frame}

%% Slide 18: Timing Diagram
\begin{frame}
\begin{center}
\includegraphics[width=0.95\textwidth]{../images_yt_video/timing_diagram.png}
\end{center}
\end{frame}

%% Slide 16: Row Hits
\begin{frame}{Row Hits (Page Hits)}
\textbf{What is a Row Hit:}
\begin{itemize}
    \item Next address is in already-open row
    \item Use only column address (10 bits)
    \item Skip row opening steps
\end{itemize}

\vspace{0.5cm}
\textbf{Benefits:}
\begin{itemize}
    \item Much faster access
    \item Can occur repeatedly on same row
    \item Systems optimized to maximize row hits
\end{itemize}
\end{frame}

%% Slide 17: Row Misses
\begin{frame}{Row Misses and Timing}
\textbf{What is a Row Miss:}
\begin{itemize}
    \item Next address in different row
    \item Must close current row, open new row
    \item Significantly slower than row hit
\end{itemize}

\vspace{0.5cm}
\textbf{Timing Parameters (clock cycles):}
\begin{itemize}
    \item CAS latency: row hit to data
    \item RAS to CAS delay: row opening time
    \item Row precharge time
    \item Row active time
\end{itemize}
\end{frame}

%% Slide 18: Timing Diagram
\begin{frame}
\begin{center}
\includegraphics[width=1.0\textwidth]{../images_yt_video/row_misses_and_timing.png}
\end{center}
\end{frame}

%% Slide 19: Multiple Banks - Parallelism
\begin{frame}{Why Multiple Banks?}
\textbf{Independent Operation:}
\begin{itemize}
    \item Each bank operates independently
    \item Multiple rows open across different banks
    \item Increases row hit probability
\end{itemize}

\vspace{0.5cm}
\textbf{Performance Impact:}
\begin{itemize}
    \item Reduced average access time
    \item Better parallel processing
\end{itemize}
\end{frame}

%% Slide 20: Bank Groups
\begin{frame}{Bank Groups Organization}
\textbf{Structure:}
\begin{itemize}
    \item 8 bank groups × 4 banks = 32 total banks
    \item Groups enable better refresh management
\end{itemize}

\vspace{0.5cm}
\textbf{Refresh Optimization:}
\begin{itemize}
    \item Refresh one bank per group
    \item Use other three banks during refresh
    \item Minimizes performance impact
\end{itemize}
\end{frame}

%% Slide 21: Burst Buffer Optimization
\begin{frame}{Burst Buffer Optimization}
\textbf{Design:}
\begin{itemize}
    \item 128-bit temporary storage buffer
    \item Column address split: 6 bits (mux) + 4 bits (buffer)
\end{itemize}

\vspace{0.5cm}
\textbf{Operation:}
\begin{itemize}
    \item Load 128 cells into buffer using 6 bits
    \item Select 8 data locations using 4 bits, cycle through 16 combinations
    \item Result: 1,024 bits accessed quickly per chip (burst length = 16)
\end{itemize}

\vspace{0.5cm}
\textbf{Benefit:} Fast sequential access with random access capability
\end{frame}

%% Slide 22: Burst Buffer Diagram
\begin{frame}
\begin{center}
\includegraphics[width=0.95\textwidth]{../images_yt_video/burst_buffers.png}
\end{center}
\end{frame}

%% Slide 23: Subarrays and Hierarchical Design
\begin{frame}{Subarrays and Hierarchical Design}
\textbf{Challenge:}
\begin{itemize}
    \item Large 65,536 × 8,192 arrays
    \item Extremely long wordlines and bitlines
\end{itemize}

\vspace{0.3cm}
\textbf{Solution:}
\begin{itemize}
    \item Subdivide into 1,024 × 1,024 subarrays
    \item Intermediate sense amplifiers per subarray
    \item Hierarchical row decoding
\end{itemize}

\vspace{0.3cm}
\textbf{Benefits:}
\begin{itemize}
    \item Shorter bitlines $\rightarrow$ smaller capacitors
    \item Shorter wordlines $\rightarrow$ reduced capacitive load
    \item Faster transistor activation
    \item Improved performance
\end{itemize}
\end{frame}

%% Slide 24: Sense Amplifier Diagram
\begin{frame}
\begin{center}
\includegraphics[width=0.95\textwidth]{../images_yt_video/sens_amp.png}
\end{center}
\end{frame}

%% Slide 25: Differential Pair Architecture
\begin{frame}{Differential Pair Architecture}
\textbf{Design:}
\begin{itemize}
    \item Two bitlines per sense amplifier
    \item Alternating rows connect to different bitlines
    \item Half bitlines active, half passive at any time
\end{itemize}

\vspace{0.3cm}
\textbf{Cross-Coupled Inverter:}
\begin{itemize}
    \item Active bitline = 1 $\rightarrow$ passive = 0
    \item Active bitline = 0 $\rightarrow$ passive = 1
    \item Creates differential pair (opposite values)
    \item Passive bitline not connected to cells
\end{itemize}

\vspace{0.3cm}
\textbf{Key Benefits:}
\begin{enumerate}
    \item Easy precharging: charge equalizes to 0.5V
    \item Noise immunity from opposite charges
    \item Reduced parasitic capacitance
\end{enumerate}
\end{frame}

%% Slide 26: Inside Sense Amplifier
\begin{frame}
\begin{center}
\includegraphics[width=0.95\textwidth]{../images_yt_video/inside_sens_amp.png}
\end{center}
\end{frame}

\end{document}